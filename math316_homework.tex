%% Standard start of a latex document
\documentclass[letterpaper,12pt]{article}
%% Always use 12pt - it is much easier to read
%% Things written after '%' sign, are ignored by the latex editor - they are how to introduce comments into your .tex source
%% Anything mathematics related should be put in between '$' signs.

%% Set some names and numbers here so we can use them below
\newcommand{\myname}{ } %%%%%%%%%%%%%%% ---------> Change this to your name
\newcommand{\mynumber}{ } %%%%%%%%%%%%%%% ---------> Change this to your student number
\newcommand{\hw}{number...} %%%%%%%%%%%%%%% --------->  set this to the homework number

%%%%%%
%% There is a bit of stuff below which you should not have to change
%%%%%%

%% AMS mathematics packages - they contain many useful fonts and symbols.
\usepackage{amsmath, amsfonts, amssymb}
\allowdisplaybreaks

%% The geometry package changes the margins to use more of the page, I suggest
%% using it because standard latex margins are chosen for articles and letters,
%% not homework.
\usepackage[paper=letterpaper,left=25mm,right=25mm,top=3cm,bottom=25mm]{geometry}
%% For details of how this package works, google the ``latex geometry documentation''.

%%
%% Fancy headers and footers - make the document look nice
\usepackage{fancyhdr} %% for details on how this work, search-engine ``fancyhdr documentation''
\pagestyle{fancy}
%%
%% The header
\lhead{Mathematics 257/316} % course name as top-left
\chead{Homework 6} % homework number in top-centre
\rhead{ \myname \\ \mynumber }
%% This is a little more complicated because we have used `` \\ '' to force a line-break between the name and number.
%%
%% The footer
\lfoot{\myname} % name on bottom-left
\cfoot{Page \thepage} % page in middle
\rfoot{\mynumber} % student number on bottom-right
%%
%% These put horizontal lines between the main text and header and footer.
\renewcommand{\headrulewidth}{0.4pt}
\renewcommand{\footrulewidth}{0.4pt}
%%%


% Some useful macros

\usepackage{amsmath,amssymb,amsthm}
\usepackage{enumerate}
\usepackage{color}

\newcommand{\ZZ}{\mathbb{Z}}
\newcommand{\FF}{\mathbb{F}}
\newcommand{\RR}{\mathbb{R}}
\newcommand{\QQ}{\mathbb{Q}}
\newcommand{\CC}{\mathbb{C}}
\newcommand{\NN}{\mathbb{N}}
\renewcommand\vec{\mathbf}

\newcommand{\st}{\text{ such that } }
\newcommand{\dee}[1]{\mathrm{d}#1}
\newcommand{\diff}[2]{ \frac{\dee{#1}}{\dee{#2}} }
\newcommand{\lt}{<}
\newcommand{\gt}{>}
\newcommand{\set}[1]{\left\{#1 \right\}}
\newcommand{\dig}[1]{\left\langle{#1}\right\rangle}
\newcommand{\closure}[1]{\overline{#1}}
\newcommand{\interior}[1]{\mathrm{int}\left(#1\right)}
\newcommand{\boundary}[1]{\delta\left(#1\right)}
\newcommand{\ceiling}[1]{\left\lceil #1 \right\rceil}

\usepackage{mdframed}

\begin{document}
% Put your answers as items in this enumerated environment and they will be automatically numbered

\begin{enumerate}[Q1.]
    \item We consider consider the following problem:
    \begin{align*}
        \frac{\partial u}{\partial x^2} = 4 \frac{\partial u}{\partial t}, \quad 0 < x < 2, t> 0 \\
        u(0,t) = 0 \text{ and } u(2,t) = 0 \text{ for } t>0 \\
        u(x,0) = 2 \sin{\frac{\pi x}{2}} + 4 \sin (2\pi x), \quad 0 \leq x \leq 2
    \end{align*}
    Solve for $u(x,t)$.
   \begin{mdframed}[leftmargin=-2cm]
        \textbf{Answer:}
        \item Assume a solution of the form: $u(x, t)=T(t) X(x)$.
        \item Then $u_{x x}=T(t) X^{\prime \prime}(x); \quad u_t=\dot{T}(t) X(x)$.
        \item Substituting these into the equation:
        \begin{align*}
            &T(t) X^{\prime \prime}(x)=4 \dot{T}(t) X(x)\\
            \Rightarrow &\frac{4 \dot{T}(t)}{T(t)}=\frac{X^{\prime \prime}(x)}{X(x)}=\lambda, \text{ for some constant } \lambda\\
            \Rightarrow & \dot{T}(t)=\frac{1}{4} \lambda T(t) \quad ; \quad X^{\prime \prime}(x)=\lambda X(x)
        \end{align*}
        \item So for the time-dependent part, its solution is $T(t)=C e^{\frac{\lambda}{4} t}, \quad C$ constant.
        \item Using the boundary conditions, we have:
            $$
            u(x, t)=T(t) X(x) \Rightarrow T(t) X(0)=0=T(t) X(2) \Rightarrow X(0)=X(2)=0
            $$
        \item So the eigenvalue problem is:
            $$
            \begin{aligned}
            & X^{\prime \prime}(x)=\lambda X(x), \quad \text { with } \\
            & X(0)=X(2)=0, \quad(0<x<2) .
            \end{aligned}
            $$
        \item \textbf{CASE 1:} $\lambda>0$ :
        \item Let $\lambda=\mu^2$, then:
            $$
            X^{\prime \prime}(x)-\mu^2 X(x)=0
            $$
        \item Since the characteristic equation: $r^2-\mu^2=0 \Rightarrow r= \pm \mu$, the general solution is :
            $$
            X(x)=A e^{\mu x}+B e^{-\mu x}
            $$
        \item Applying the boundary conditions:
            $$
            \begin{aligned}
            X(0) & =A+B=0 \Rightarrow B=-A \\
            \Rightarrow X(x) & =A\left(e^{\mu x}-e^{-\mu x}\right)=2 A \sinh (\mu x) \\
            X(2) & =2 A \sinh (2 \mu)=0
            \end{aligned}
            $$
        \item Since $\sinh (2 \mu)>0$ for $\mu>0$, it must be true that $A=0=B$, leading to a trivial solution.
        \item \textbf{CASE 2:} $\lambda=0$ :
        \item The equation becomes:
            $$
            X^{\prime \prime}(x)=0 \Rightarrow X^{\prime}(x)=A ; \quad X(x)=A x+B
            $$
        \item Applying the boundary conditions:
            $$
            \begin{aligned}
            & X(0)=B=0 \\
            & X(2)=2 A=0 \Rightarrow A=0
            \end{aligned}
            $$
            $$
            A=B=0 \text { gives a trivial solution. }
            $$
        \item \textbf{CASE 3:} $\lambda<0$ :
        \item Let $\lambda=-\mu^2$, then:
            $$
            X^{\prime \prime}+\mu^2 X=0
            $$
        \item Since the characteristic equation: $r^2+\mu^2=0 \Rightarrow r= \pm i \mu$, the general solution is:
            $$
            X(x)=A \cos (\mu x)+B \sin (\mu x)
            $$
        \item Applying the boundary conditions:
            $$
            \begin{aligned}
            X(0) & =A=0 \\
            \Rightarrow \quad X(x) & =B \sin (\mu x) \\
            X(2) & =B \sin (2 \mu)=0
            \end{aligned}
            $$
        \item For a non-trivial solution $(B \neq 0)$, we need:
            $$
            \sin (2 \mu)=0 \quad \Rightarrow \quad 2 \mu=n \pi, \quad n=1,2,3, \ldots
            $$
        \item Thus, nontrivial solution exists for
            $$
            \mu=\frac{n \pi}{2} \text { and } \lambda_n=-\left(\frac{n \pi}{2}\right)^2
            $$
        \item The corresponding eigenfunction are:
            $$
            X_n(x)=B_n \sin \left(\frac{n \pi x}{2}\right), \quad n=1,2,3, \ldots
            $$
        \item Since $B_n$ gets absorbed into the Fourier coefficients later, without the loss of generality, we can choose:
            $$
            X_n(x)=\sin \left(\frac{n \pi x}{2}\right), \quad n=1,2,3, \ldots
            $$
        \item So the solution will have the form:
            $$
            u_n(x, t)=\sin \left(\frac{n \pi x}{2}\right) e^{-\left(\frac{n \pi}{4}\right)^2 t}, n=1,2,3, \ldots
            $$
        \item Since the equation is linear, the most general solution is of the form:
            $$
            u(x, t)=\sum_{n=1}^{\infty} A_n \sin \left(\frac{n \pi x}{2}\right) e^{-\left(\frac{n \pi}{4}\right)^2 t} \text {, for some coefficients } A_n, n=1,2,3, \ldots
            $$
        \item Using the initial condition $u(x, 0)=2 \sin \left(\frac{\pi x}{2}\right)-\sin (\pi x)+4 \sin (2 \pi x)$ :
            $$
            2 \sin \left(\frac{\pi x}{2}\right)-\sin (\pi x)+4 \sin (2 \pi x)=\sum_{n=1}^{\infty} A_n \sin \left(\frac{n \pi x}{2}\right)
            $$
        \item Note that $\sin \left(\frac{\pi x}{2}\right)$ corresponds to $n=1, \sin (\pi x)=\sin \left(\frac{2 \pi x}{2}\right)$ corresponds to $n=2$, and $\sin (2 \pi x)=\sin \left(\frac{4 \pi x}{2}\right)$ corresponds to $n=4$.
        \item So the Fourier coefficients are:
            $$
            A_1=2, \quad A_2=-1, A_4=4 \text {, and } A_n=0 \text { for all the other } n \text {. }
            $$
        \item Thus, the final solution to this equation is:
            \begin{align*}
            u(x, t)&=2 \sin \left(\frac{\pi x}{2}\right) e^{-\frac{\pi^2}{16} t}-\sin (\pi x) e^{-\frac{4 \pi^2}{16} t}+4 \sin (2 \pi x) e^{-\frac{16 \pi^2}{16} t} \\
            & =2 \sin \left(\frac{\pi x}{2}\right) e^{-\frac{\pi^2}{16} t}-\sin (\pi x) e^{-\frac{\pi^2}{4} t}+4 \sin (2 \pi x) e^{-\pi^2 t}
            \end{align*}
        \end{mdframed}
\end{enumerate}
\pagebreak
\begin{enumerate}[Q2.]
    \item We consider the problem of heat diffusion in a metal rod of length $L = 1$.
    \begin{align*}
        \frac{\partial u}{\partial t} - \frac{\partial ^2u}{\partial x^2} = 0, \text{ for } 0 < x < 1 \text{ and } t \geq 0 \\
        u(0,t) = 0 \text{ and } \frac{\partial u}{\partial x} (1,t) = 0, \text{ for } t > 0 \\
        u(x,0) = x(1-x) \text{ for } 0 \leq x \leq 1
    \end{align*}
    Determine the temperature $u(x,t)$ using the method of separation of variables.
    \setcounter{equation}{0} 
    \begin{mdframed}[leftmargin=-2cm]
            \textbf{Answer:}
            \item Assume a solution of the form: $u(x, t)=X(x) T(t)$
            \item Then $u_t=X(x) \dot{T}(t); \quad u_{x x}=X^{\prime \prime}(x) T(t)$
            \item Substituting these into the PDE:
                $$
                \begin{array}{ll} 
                & X(x) \dot{T}(t)-X^{\prime \prime}(x) T(t)=0 \\
                \Rightarrow & X(x) \dot{T}(t)=X^{\prime \prime}(x) T(t)
                \end{array}
                $$
                $$
                \Rightarrow \frac{\dot{T}(t)}{T(t)}=\frac{X^{\prime \prime}(x)}{X(x)}=\lambda, \quad \text { for some constant } \lambda
                $$
            \item So the time-dependent ODE: $\dot{T}(t)=\lambda T(t) \Rightarrow T(t)=e^{\lambda t}$
            \item And the spatial ODE: $X^{\prime \prime}(x)-\lambda X(x)=0$
            \item Using the boundary conditions, we have:
                $$
                \left.\begin{array}{l}
                u(x, t)=X(x) T(t) \Rightarrow u(0, t)=X(0) T(t)=0 \\
                u_x(x, t)=X^{\prime}(x) T(t) \Rightarrow u_x(1, t)=X^{\prime}(1) T(t)=0
                \end{array}\right\} \Rightarrow X(0)=0=X^{\prime}(1)
                $$
            \item So the eigenvalue problem is:
                $$
                \begin{aligned}
                & X^{\prime}(x)-\lambda X(x)=0, \quad \text { with } \\
                & X(0)=0, \quad X^{\prime}(1)=0
                \end{aligned}
                $$
            \item \textbf{CASE 1:} $\lambda>0$ :
            \item Let $\lambda=\mu^2$, then:
                $X^{\prime}(x)-\mu^2 X(x)=0$, so the general solution is
                $$
                X(x)=A e^{\mu x}+B e^{-\mu x}
                $$
            \item Applying the boundary conditions:
                $$
                \begin{aligned}
                X(0) & =A+B=0 \Rightarrow B=-A \\
                \Rightarrow X(x) & =A\left(e^{\mu x}-e^{-\mu x}\right)=2 A \sinh (\mu x) \Rightarrow X^{\prime}(x)=2 \mu A \cosh (\mu x) \\
                X^{\prime}(1) & =2 \mu A \cosh (\mu)
                \end{aligned}
                $$
            \item Since $\cosh (\mu)>0$ for $\mu>0$, it must be true that $A=0=B$, leading to a trivial solution. 
            \item \textbf{CASE 2:} $\lambda=0$ :
            \item The equation becomes:
                $$
                X^{\prime \prime}(x)=0 \Rightarrow X^{\prime}(x)=A ; X(x)=A x+B
                $$
            \item Applying the boundary conditions:
                $$
                \begin{aligned}
                & X(0)=B=0 \\
                & X^{\prime}(1)=A=0
                \end{aligned}
                $$
            \item $A=B=0$ gives a trivial solution.
            \item \textbf{CASE 3:} $\lambda<0$ :
            \item Let $\lambda=-\mu^2$, then:
                $$
                X^{\prime \prime}+\mu^2 X=0
                $$
            \item Since the characteristic equation: $r^2+\mu^2=0 \Rightarrow r= \pm i \mu$, the general solution is:
                $$
                X(x)=A \cos (\mu x)+B \sin (\mu x)
                $$
            \item Applying the boundary conditions:
                \begin{align*}
                X(0)&=A=0\\
                \Rightarrow X(x)&=B \sin (\mu x) \Rightarrow X^{\prime}(x)=\mu B \cos (\mu x) \\
                X^{\prime}(1)&=\mu B \cos (\mu)=0
                \end{align*}
            \item For a nontrivial solution $(B \neq 0)$, we require:
                \begin{align*}
                \cos (\mu)=0 \Rightarrow \mu=\frac{\pi}{2}+n \pi, \quad n=0,1,2,3, \ldots
                \end{align*}
            \item Thus, a nontrivial solution exists for
                \begin{align*}
                \mu=\frac{(2n+1) \pi}{2}, \quad \lambda=-\left[\frac{(2 n+1) \pi}{2}\right]^{2}
                \end{align*}
            \item The corresponding eigenfunctions are:
                \begin{align*}
                X_{n}(x)=B_{n} \sin \left(\frac{(2 n+1) \pi}{2} x\right), \quad n=0,1,2,3, \ldots
                \end{align*}
            \item So the solution will have the form:
                \begin{align*}
                u_{n}(x, t)=\sin \left(\frac{(2 n+1) \pi}{2} x\right) e^{-\frac{(2 n+1)^{2} \pi^{2}}{4} t}, \quad n=0,1,2,3, \ldots
                \end{align*}
           \item Since the equation is linear, the most general solution is of the form:
                \begin{align*}
                u(x, t)=\sum_{n=0}^{\infty} A_{n} \sin \left(\frac{(2 n+1) \pi}{2} x\right) e^{-\frac{(22+1)^{2} \pi^{2}}{4} t}, \quad \text { for some constant } A_{n}, n=0,1,2,3, \ldots
                \end{align*}
            \item Applying the initial condition $u(x, 0)=x(1-x)$ :
                \begin{align}
                \sum_{n=0}^{\infty} A_{n} \sin \left(\frac{(2 n+1) \pi}{2} x\right)=x(1-x)
                \end{align}
            \item Because $\sin \left(\frac{(2 n+1) \pi}{2} x\right)$ is orthogonal over the interval $[0,1]$, we multiply both sides of $(1)$ by $\sin \left(\frac{(2 m+1) \pi}{2} x\right)$ and integrate over $[0,1]$:
            \begin{align}
                \sum_{n=0}^{\infty} A_{n} \int_{0}^{1} \sin \left(\frac{(2 n+1) \pi}{2} x\right) \sin \left(\frac{(2 m+1) \pi}{2} x\right) d x=\int_{0}^{1}\left(x-x^{2}\right) \sin \left(\frac{(2 m+1) \pi}{2} x\right) d x
            \end{align}
            \item The orthogonal property tells us that:
                \begin{align*}
                \int_{0}^{1} \sin \left(\frac{(2 n+1) \pi}{2} x\right) \sin \left(\frac{(2 m+1) \pi}{2} x\right) d x= \begin{cases}0, & \text { if } n \neq m \\ \frac{1}{2}, & \text { if } n=m .\end{cases}
                \end{align*}
            \item So (2) becomes:
                \begin{align*}
                A_{m} \frac{1}{2}=\int_{0}^{1}\left(x-x^{2}\right) \sin \left(\frac{(2 m+1) \pi}{2} x\right) d x
                \end{align*}
            \item Thus, the coefficient is
                \begin{align*}
                A_{m} & =2 \int_{0}^{1}\left(x-x^{2}\right) \sin \left(\frac{(2 m+1) \pi}{2} x\right) d x \\
                & =2\left[\int_{0}^{1} x \sin \left(\frac{(2 m+1) \pi}{2} x\right) d x-\int_{0}^{1} x^{2} \sin \left(\frac{(2 m+1) \pi}{2} x\right) d x\right]
                \end{align*}
            \item Let $\mu=\frac{(2 m+1) \pi}{2}$ for simplicity.
            \item Simplifying $A_{m}$ using integration by parts:
            \item For $\int_{0}^{1} x \sin (\mu x) d x $ term:
            \item Take $=x, \quad dv = \sin (\mu x) dx$
            \item $\Rightarrow d u=d x, \quad v=-\frac{1}{\mu} \cos (\mu x)$
            \item $\Rightarrow \int_{0}^{1} x \sin (\mu x) d x =\left[-\frac{x}{\mu} \cos (\mu x)\right]_{0}^{1}+\frac{1}{\mu} \int_{0}^{1} \cos (\mu x) d x$
            \item $\qquad \qquad \qquad \qquad =-\frac{1}{\mu} \cos (\mu)+\frac{1}{\mu}\left[\frac{1}{\mu} \sin (\mu x)\right]_{0}^{1}$
            \item $\qquad \qquad \qquad \qquad =-\frac{\cos (\mu)}{\mu}+\frac{\sin (\mu)}{\mu^{2}}$
            \item For $\int_{0}^{1} x^{2} \sin (\mu x) d x$ term:
            \item Take $u=x^{2}, \quad d v=\sin (\mu x) d x$
            \item $\Rightarrow d u=2 x d x, v=-\frac{1}{\mu} \cos (\mu x)$
            \item $\Rightarrow \int_{0}^{1} x^{2} \sin (\mu x) d x=\left[-\frac{x^{2}}{\mu} \cos (\mu x)\right]_{0}^{1}+\frac{2}{\mu} \int_{0}^{1} x \cos (\mu x) d x$
            \item $\qquad \qquad \qquad \qquad =-\frac{\cos (\mu)}{\mu}+\frac{2}{\mu} \int_{0}^{1} x \cos (\mu x) d x$
            \item Take $u=x, \quad d v=\cos (\mu x) d x$
            \item $\qquad \qquad \quad \Rightarrow d u=d x, \quad v=\frac{1}{\mu} \sin (\mu x)$
            \item \begin{align*}
            \begin{aligned}
            \Rightarrow-\frac{\cos (\mu)}{\mu}+\frac{2}{\mu} \int_{0}^{1} x \cos (\mu x) d x & =-\frac{\cos (\mu)}{\mu}+\frac{2}{\mu}\left\{\left[\frac{x}{\mu} \sin (\mu x)\right]_{0}^{1}-\frac{1}{\mu} \int_{0}^{1} \sin (\mu x) d x\right\} \\
            & =-\frac{\cos (\mu)}{\mu}+\frac{2}{\mu}\left[\frac{\sin (\mu)}{\mu}+\frac{1}{\mu}\left[\frac{1}{\mu} \cos (\mu x)\right]_{0}^{1}\right] \\
            & =-\frac{\cos (\mu)}{\mu}+\frac{2}{\mu}\left[\frac{\sin (\mu)}{\mu}+\frac{\cos (\mu)-1}{\mu^{2}}\right] \\
            & =-\frac{\cos (\mu)}{\mu}+\frac{2 \sin (\mu)}{\mu^{2}}+\frac{2(\cos (\mu)-1)}{\mu^{3}}
            \end{aligned}
            \end{align*}
            \item So the coefficient is:
            \begin{align*}
            \begin{aligned}
            A_{m} & =2\left[\int_{0}^{1} x \sin \left(\frac{(2 m+1) \pi}{2} x\right) d x-\int_{0}^{1} x^{2} \sin \left(\frac{(2 m+1) \pi}{2} x\right) d x\right] \\
            & =2\left[-\frac{\cos (\mu)}{\mu}+\frac{\sin (\mu)}{\mu^{2}}+\frac{\cos (\mu)}{\mu}-\frac{2 \sin (\mu)}{\mu^{2}}-\frac{2(\cos \mu-1)}{\mu^{3}}\right] \\
            & =2\left[-\frac{\sin \mu}{\mu^{2}}-\frac{2(\cos \mu-1)}{\mu^{3}}\right] \\
            & =2\left[-\frac{\sin \left(\frac{(2 m+1) \pi}{2}\right)}{\left(\frac{(2 m+1) \pi}{2}\right)^{2}}-\frac{2\left[\cos \left(\frac{(2 m+1) \pi}{2}\right)-1\right]}{\left(\frac{(2 m+1) \pi}{2}\right)^{3}}\right]
            \end{aligned}
            \end{align*}
            \item Note that $\sin \left(\frac{(2 m+1) \pi}{2}\right)=(-1)^{m}$ and $\cos \left(\frac{(2 m+1) \pi}{2}\right)=0$, for $m=0,1,2, \ldots$
            So 
            \begin{align*}
            \begin{aligned}
                A_{m}&=2\left[-\frac{4(-1)^{m}}{(2 m+1)^{2} \pi^{2}}-\frac{16(-1)}{(2 m+1)^{3} \pi^{3}}\right]\\
            & =\frac{32}{(2 m+1)^{3} \pi^{3}}-\frac{8(-1)^{m}}{(2 m+1)^{2} \pi^{2}} \\
            & =\frac{8}{(2 m+1)^{2} \pi^{2}}\left[\frac{4}{(2 m+1) \pi}-(-1)^{m}\right], \qquad m=0,1,2, \ldots
            \end{aligned}
            \end{align*}
            \item And the solution is:
            \begin{align*}
            u(x, t)=\sum_{n=0}^{\infty} \frac{8}{(2 n+1)^{2} \pi^{2}}\left[\frac{4}{(2 n+1) \pi}-(-1)^{n}\right] \sin \left(\frac{(2 n+1) \pi}{2} x\right) e^{-\frac{(2 n+1)^{2} \pi^{2}}{4} t}
\end{align*}
            
        \end{mdframed}
    \end{enumerate}

\end{document}